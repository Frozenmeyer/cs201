\documentclass[12pt]{article}
\input{preamble.tex}

% CHANGE ZZZ to the number of the course you are submitting to and NN to the
% number of the homework. \bfseries is there to make the typeface bold

\title{\bfseries CS 201 Homework 01}
\author{Leif O'nye King}
\date{9/1/2021}

%% -----------------------------------
%% N E E D   H E L P ? ---------------
%% -----------------------------------
%% https://en.wikibooks.org/wiki/LaTeX
%% -----------------------------------

\begin{document}

\maketitle

%%%%%%%%%%%%%%%%%%%%%%%%%%%%%%%%%%%%%%%%%%%%%%%%%%%%%%%%%%%%%%%%%%%%%%%
% LINKS SECTION %%%%%%%%%%%%%%%%%%%%%%%%%%%%%%%%%%%%%%%%%%%%%%%%%%%%%%%
%%%%%%%%%%%%%%%%%%%%%%%%%%%%%%%%%%%%%%%%%%%%%%%%%%%%%%%%%%%%%%%%%%%%%%%

\begin{itemize}
\item Repository Link:
% This is where you will put a link to your GitHub project
\url{https://github.com/Frozenmeyer/cs201.git}

% This is where you will put a link to your GitHub project commits
\item Git Commits:
\url{https://github.com/Frozenmeyer/cs201.git}

% PLEASE NOTE THE AMOUNT OF TIME THIS HOMEWORK TOOK

\item This homework took approximately XX hours to complete.
\end{itemize}




%%%%%%%%%%%%%%%%%%%%%%%%%%%%%%%%%%%%%%%%%%%%%%%%%%%%%%%%%%%%%%%%%%%%%%%
% DESIGN SECTION %%%%%%%%%%%%%%%%%%%%%%%%%%%%%%%%%%%%%%%%%%%%%%%%%%%%%%
%%%%%%%%%%%%%%%%%%%%%%%%%%%%%%%%%%%%%%%%%%%%%%%%%%%%%%%%%%%%%%%%%%%%%%%

\section{Design}

Getting each of the individual applications was easy. I already had Github, Visual Studio and VS code were easy to get. Getting all of these applications to play together was a nightmare and I'm uncertain I've  done it correctly even now.




%%%%%%%%%%%%%%%%%%%%%%%%%%%%%%%%%%%%%%%%%%%%%%%%%%%%%%%%%%%%%%%%%%%%%%%
% POST MORTEM SECTION %%%%%%%%%%%%%%%%%%%%%%%%%%%%%%%%%%%%%%%%%%%%%%%%%
%%%%%%%%%%%%%%%%%%%%%%%%%%%%%%%%%%%%%%%%%%%%%%%%%%%%%%%%%%%%%%%%%%%%%%%

\section{Post Mortem}

A lot went wrong for quite a while. Linking and learning how these different applications work is realistically still a work in progress. I think I've cobbled together enough work arounds to submit and test code




%%%%%%%%%%%%%%%%%%%%%%%%%%%%%%%%%%%%%%%%%%%%%%%%%%%%%%%%%%%%%%%%%%%%%%%
% QUESTION ANSWER SECTION %%%%%%%%%%%%%%%%%%%%%%%%%%%%%%%%%%%%%%%%%%%%%
%%%%%%%%%%%%%%%%%%%%%%%%%%%%%%%%%%%%%%%%%%%%%%%%%%%%%%%%%%%%%%%%%%%%%%%

\section{Answers to Questions}

In this section, you will write the answers to the questions in the homework assignment.

\begin{enumerate}
    \item Software can be very versatile problem solving. Trained neural networks can be used for all sorts of pattern recognition specifically early detection of cancerous cells.\ldots
    \item Anyone can be a software developer, but most of the time they are pragmatic and probably tired. If you're between jobs you'll likely wear a suit, otherwise a T-shirt. \ldots
    \item Ram: 4 GB, Disk: 256 GB, Main Memory: ??\ldots
    \item #1 Video Games, #2 Eventually Artificial Intelligence, #3 Movies and other streaming, #4 Software developement, #5 Web developement. I'm most interested in Software / Webdevelopement as a career, but I love working on games so that may be a side project of mine. Additionally AI developements will be very interesting to follow up on and have personal knowledge of, although I doubt I'll be of much use in the field.\ldots
\end{enumerate}




%%%%%%%%%%%%%%%%%%%%%%%%%%%%%%%%%%%%%%%%%%%%%%%%%%%%%%%%%%%%%%%%%%%%%%%
% PROGRAM 1 SECTION %%%%%%%%%%%%%%%%%%%%%%%%%%%%%%%%%%%%%%%%%%%%%%%%%%%
%%%%%%%%%%%%%%%%%%%%%%%%%%%%%%%%%%%%%%%%%%%%%%%%%%%%%%%%%%%%%%%%%%%%%%%

\section{
    #include <iostream>
    
    int main(){
        std::cout << "Hellow World!" << std::endl;
    }}

%%%%%%%%%%%%%%%%%%%%%%%%%%%%%%%%%%%%%%%%%%%%%%%%%%%%%%%%%%%%
%% SAMPLE OUTPUT / SCREENSHOT %%%%%%%%%%%%%%%%%%%%%%%%%%%%%%
%%%%%%%%%%%%%%%%%%%%%%%%%%%%%%%%%%%%%%%%%%%%%%%%%%%%%%%%%%%%
\subsection{Sample Output/Screenshot}

% We are going to color the output BLUE, and then set it back to BLACK
% The lstlisting environment documentation may be found
% https://en.wikibooks.org/wiki/LaTeX/Source_Code_Listings
% http://tug.ctan.org/macros/latex/contrib/listings/listings.pdf
\lstset{language=, caption=Sample Program Output, label=lst:output}
\color{blue}
\begin{lstlisting}
Hello, world
PUT YOUR OUTPUT HERE
\end{lstlisting}
\color{black}

Or you can put a screenshot:

% Or a screenshot
\begin{center}
\includegraphics[width=0.5\textwidth]{screenshot.png}
\end{center}

%%%%%%%%%%%%%%%%%%%%%%%%%%%%%%%%%%%%%%%%%%%%%%%%%%%%%%%%%%%%
%% GIT COMMIT MESSAGES %%%%%%%%%%%%%%%%%%%%%%%%%%%%%%%%%%%%%
%%%%%%%%%%%%%%%%%%%%%%%%%%%%%%%%%%%%%%%%%%%%%%%%%%%%%%%%%%%%
\subsection{Git Commit Messages}

\begin{centering}
\begin{tabularx}{\linewidth}{c X}
\thead{Date} & \thead{Message} \\
\hline
2019-11-07 & \text{fix crash where maxhistorylines was 0 in hflog} \\
2019-11-07 & \text{Add pitch() method to TImage} \\
\hline
\end{tabularx}
\end{centering}


%%%%%%%%%%%%%%%%%%%%%%%%%%%%%%%%%%%%%%%%%%%%%%%%%%%%%%%%%%%%
%% SOURCE CODE %%%%%%%%%%%%%%%%%%%%%%%%%%%%%%%%%%%%%%%%%%%%%
%%%%%%%%%%%%%%%%%%%%%%%%%%%%%%%%%%%%%%%%%%%%%%%%%%%%%%%%%%%%
\subsection{Source Code}

% You could also \inputminted{c++}{myfile.cpp}
% or just put it here in the minted environment
\begin{minted}{c++}
#include <iostream>
#include "triangle.hpp"

using std::cout;
using std::endl;

int main(int argc, char **argv)
{
    cout << "Hello, World\n";
    Triangle t;
    t.print();
    return 0;
}
\end{minted}

\subsection{Triangle Header}

\inputminted{c++}{triangle.hpp}

\subsection{Triangle Source}

\inputminted{c++}{triangle.cpp}




%%%%%%%%%%%%%%%%%%%%%%%%%%%%%%%%%%%%%%%%%%%%%%%%%%%%%%%%%%%%%%%%%%%%%%%
% PROGRAM 2 SECTION %%%%%%%%%%%%%%%%%%%%%%%%%%%%%%%%%%%%%%%%%%%%%%%%%%%
%%%%%%%%%%%%%%%%%%%%%%%%%%%%%%%%%%%%%%%%%%%%%%%%%%%%%%%%%%%%%%%%%%%%%%%

\section{Program 2}


\subsection{Sample Output / Screenshot}
% \begin{center}
% \includegraphics[width=0.5\textwidth]{screenshot.png}
% \end{center}


\subsection{Git Commit Messages}

\begin{centering}
\begin{tabularx}{\linewidth}{c X}
\thead{Date} & \thead{Message} \\
\hline
2019-11-07 & fix crash where maxhistorylines was 0 in hflog \\
2019-11-07 & Add pitch() method to TImage \\
\hline
\end{tabularx}
\end{centering}


\subsection{Source Code}
% \inputminted{c++}{main.cpp}


%\subsection{File.hpp}
% \inputminted{c++}{File.hpp}


%\subsection{File.cpp}
% \inputminted{c++}{File.cpp}




%%%%%%%%%%%%%%%%%%%%%%%%%%%%%%%%%%%%%%%%%%%%%%%%%%%%%%%%%%%%%%%%%%%%%%%
% PROGRAM 3 SECTION %%%%%%%%%%%%%%%%%%%%%%%%%%%%%%%%%%%%%%%%%%%%%%%%%%%
%%%%%%%%%%%%%%%%%%%%%%%%%%%%%%%%%%%%%%%%%%%%%%%%%%%%%%%%%%%%%%%%%%%%%%%

\section{Program 3}

\subsection{Sample Output / Screenshot}
% \begin{center}
% \includegraphics[width=0.5\textwidth]{screenshot.png}
% \end{center}

\subsection{GitHub Commit Messages}

\begin{centering}
\begin{tabularx}{\linewidth}{c X}
\thead{Date} & \thead{Message} \\
\hline
2019-11-07 & \text{fix crash where maxhistorylines was 0 in hflog} \\
2019-11-07 & \text{Add pitch() method to TImage} \\
\hline
\end{tabularx}
\end{centering}

\subsection{Source Code}
% \inputminted{c++}{main.cpp}


%\subsection{File.hpp}
% \inputminted{c++}{File.hpp}


%\subsection{File.cpp}
% \inputminted{c++}{File.cpp}



\end{document}
